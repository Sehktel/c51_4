\documentclass{article}
\usepackage[utf8]{inputenc}
\usepackage[russian]{babel}
\usepackage{amsmath}
\usepackage{graphicx}
\usepackage{listings}

\title{Расширенная документация по бинарному интерфейсу AT89S4051}
\author{Sehktel}
\date{\today}

\begin{document}

\maketitle

\section{Введение}
Настоящий документ представляет детальное описание бинарного интерфейса микроконтроллера AT89S4051, охватывающее архитектурные, памятные и вычислительные особенности целевой платформы.

\section{Архитектурный анализ}
\subsection{Вычислительные характеристики}
Микроконтроллер AT89S4051 представляет собой 8-битную микроЭВМ с гарвардской архитектурой, оптимизированную для встраиваемых систем с ограниченными вычислительными ресурсами:

\begin{itemize}
    \item \textbf{Программная память}: 4 КБ (4096 байт) ROM
    \item \textbf{Оперативная память}: 128 байт RAM
    \item \textbf{Регистровый файл}: 32 регистра общего назначения
    \item \textbf{Аппаратный стек}: Глубина 32 уровня
    \item \textbf{Тактовая частота}: До 24 МГц
    \item \textbf{Разрядность}: 
        \begin{itemize}
            \item Инструкции: 8 бит
            \item Шина данных: 8 бит
        \end{itemize}
\end{itemize}

\subsection{Математическая модель адресации}
Адресное пространство микроконтроллера можно представить математической моделью:

\[ 
\begin{cases} 
ROM: [0x0000, 0x0FFF] \\
RAM: [0x0000, 0x007F] \\
SFR: [0x80, 0xFF]
\end{cases}
\]

\section{Регистровая архитектура}
\subsection{Классификация регистров}
Регистры микроконтроллера разделены на категории с учетом их функционального назначения:

\begin{itemize}
    \item \textbf{Основные вычислительные}:
        \begin{itemize}
            \item Аккумулятор (A): Примарный регистр арифметических операций
            \item Вспомогательный регистр (B): Используется в умножении/делении
        \end{itemize}
    
    \item \textbf{Адресные}:
        \begin{itemize}
            \item DPTR (16-битный указатель данных)
            \item SP (указатель стека)
        \end{itemize}
    
    \item \textbf{Статусные}:
        \begin{itemize}
            \item PSW (Program Status Word): Флаги состояния
        \end{itemize}
    
    \item \textbf{Регистры общего назначения}:
        \begin{itemize}
            \item Сохраняемые: R0-R3
            \item Временные: R4-R7
        \end{itemize}
\end{itemize}

\section{Соглашения о вызовах функций}
\subsection{Передача параметров}
Механизм передачи параметров оптимизирован под ограниченные ресурсы:

\begin{itemize}
    \item До 4 аргументов через регистры R0-R3
    \item Возвращаемое значение - в аккумуляторе A
    \item Максимальная длина функции: 256 байт
\end{itemize}

\section{Система прерываний}
\subsection{Источники и приоритезация}
Прерывания реализованы с аппаратной приоритизацией:

\begin{itemize}
    \item INT0: Внешнее прерывание 0
    \item INT1: Внешнее прерывание 1
    \item Timer0: Прерывание таймера 0
    \item Timer1: Прерывание таймера 1
    \item Serial: Прерывание последовательного порта
\end{itemize}

\section{Типизация данных}
\subsection{Примитивные типы}
\begin{itemize}
    \item \textbf{Беззнаковый байт}: 8 бит $[0, 255]$
    \item \textbf{Знаковый байт}: 8 бит $[-128, 127]$
    \item \textbf{Беззнаковое слово}: 16 бит $[0, 65535]$
    \item \textbf{Указатель}: 16 бит, выравнивание 1 байт
\end{itemize}

\section{Заключение}
Представленная спецификация ABI демонстрирует высокоэффективную архитектуру микроконтроллера AT89S4051, оптимизированную для встраиваемых систем с жесткими ресурсными ограничениями.

\end{document}